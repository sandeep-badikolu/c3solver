\documentclass[12pt, letterpaper]{article}
\usepackage[utf8]{inputenc}

\author{
Diego Iván Garza Tristán\\
1678477
\and
Moisés de Jesús Gálvez Rivera\\
1791853
\and
Andrea Amador Burgueño\\
1791890
\and
Sergio Andrés Muñoz Pinedo\\
1669194
\and
César Javier\\
\\
Universidad Autónoma de Nuevo León \\
Facultad de Ingeniería Mecánica y Eléctrica \\
}

\title{C3: Robot solucionador de cubos de Rubik.}
\date{19 de Mayo 2017}

\begin{document}
\maketitle
\newpage
\begin{abstract}
Construcción de un robot/máquina que resuelve cubos de Rubik 3x3x3 mediante el método de Kociemba, escrito en Arduino y Python.
\end{abstract}

\section{Introducción.}
\subsection{Historia del cubo de Rubik.}
stark contrasts of the human condition: bewildering problems and triumphant intelligence; simplicity and complexity; stability and dynamism; order and chaos.For this magic object to become the most popular toy in history a few chance meetings had to take place.  The first Magic Cubes
The first Magic Cubes

En 1974, un joven profesor de Arquitectura en Budapest (Hungría) llamado Erno Rubik creó un objeto que no estaba supuesto a ser posible.
Su cubo sólido se retorcía y giraba, y aún así no se quebrara o se caía a pedazos.Con stickers de varios colores en los lados, el cubo fue deshecho y así resultó el primer "Cubo de Rubik". Le tomó más de un mes a Erno para poder encontrar la solución a su propio puzzle. Más lo que no se esperaba era que el "Rubik's cube" se convertiría en el juguete más vendido de todo el mundo. Como un maestro, Erno siempre estaba buscando por nuevas, más emocionantes maneras de presentar información, por lo que él usó el primer modelo de su cubo para ayudarle a explicar a sus estudiantes acerca de relaciones espaciales. Erno siempre ha visto al cubo primeramente como un objeto de arte, una escultura móvil que simbolizaba contrastes stark 
As with many of the world’s greatest inventions it did not have an easy birth. After presenting his prototype to his students and friends Erno began to realise the potential of his cube. The next step was to get it manufactured. The first cubes were made and distributed in Hungary by Politechnika. These early Cubes, marketed as “Magic Cubes” (or “Buvos Kocka”), were twice the weight of the ones available later. In the 70’s Hungary was part of the Communist regime behind the Iron Curtain, and any imports or exports where tightly controlled. How was Erno’s invention, that had become a major success in Hungary, going to make it into the hands of every child of the 80’s? The first step in the Rubik’s Cube’s battle to worldwide recognition was to get out of Hungary. This was accomplished partly by the enchanted mathematicians who took the Cubes to international conferences and partly by an expat Hungarian entrepreneur who took the cube to the Nuremberg Toy Fair in 1979. It was at there that Tom Kremer, a toy specialist, agreed to sell it to the rest of the world. Tom’s unrelenting belief in the Cube finally resulted in the Ideal Toy Company taking on distribution of the “Magic Cube”. Ideal Toy’s executives thought that the name had overtones of witchcraft and after going through several possibilities the name: “Rubik’s Cube” was decided on, and the icon was born. In the time since its international launch in 1980 an estimated 350 million Rubik’s Cubes have been sold. Approximately one in seven people alive have played with a Rubik’s Cube. This little six color cube has gone on to represent a decade. It has started art movements (Rubik Cubism); pop videos, Hollywood movies and even had its own TV show; it has come to represent both genius and confusion; it has birthed a sport (Speedcubing); and it has even been into space. The beauty of the Rubik’s Cube is that when you look at a scrambled one, you know exactly what you need to do without instruction. Yet without instruction it is almost impossible to solve, making it one of the most infuriating and engaging inventions ever conceived.

\end{document}